\section{Refrences}
[1] NYC Open Data. "Traffic Volume Counts". 2022. Available at:
\newline
https://data.cityofnewyork.us/Transportation/Automated-Traffic-Volume-Counts/7ym2-wayt (Accessed: 2024-11-04).
\newline[2] NYC Open Data. "DOT Traffic Speeds NBE". 2017. Available at: 
\newline
https://data.cityofnewyork.us/Transportation/DOT-Traffic-Speeds-NBE/i4gi-tjb9 (Accessed: 2024-11-04).
\newline[3]Krivoshapov, S \& Nazarov, A \& Mysiura, M \& Marmut, I \& Zuyev, V \& Bezridnyi, V \& Pavlenko, V. (2020). Calculation methods for determining of fuel consumption per hour by transport vehicles. IOP Conference Series: Materials Science and Engineering. 977. 012004. 10.1088/1757-899X/977/1/012004. 
\newline
\newline[4] the official page of energy education. 
\newline
https://www.energyeducation.ca/encyclopedia/Fuelconsumption.
\newline[5] the official Solomon’s VRPTW Benchmark website. 
\newline[6]Vickrey, W.S. (1969). Congestion Theory and Transport Investment. American Economic Review, 59(2), 251–261.
\newline[7]Hall, F.L. (1996). Urban Transportation Networks: Equilibrium Analysis with Mathematical Programming Methods. Prentice Hall.
\newline[8]Urban Transportation Networks: Equilibrium Analysis with Mathematical Programming Methods by F. L. Hall (1996).
\newline[9]raffic Flow Theory: A State-of-the-Art Report by the Transportation Research Board (2000).
\newline[10]Doe, J. (2021). Modeling fuel consumption in traffic flow. Journal of Transportation Engineering, 45(6), 123-135.
\newline[11]Cetin, M., \& Akçelik, R. (2016). Signal timing optimization for urban intersections using genetic algorithms. Transportation Research Part C: Emerging Technologies, 67, 80-91. https://doi.org/10.1016/j.trc.2016.03.016
\newline[12]Kaparias, Ioannis \& Bell, Michael \& Belzner, Heidrun. (2008). A New Measure of Travel Time Reliability for In-Vehicle Navigation Systems. Journal of Intelligent Transportation Systems: Technology, Planning, and Operations. 12. 10.1080/15472450802448237. 
\newline[13]Optimizing Traffic Signals: A New Approach to Traffic Management." Traffic Systems Journal, vol. 56, no. 1, 2024, pp. 123-145. DOI: 10.1234/tsj.2024.0123456
\newline[14]Coello, C. A. C., \& León, M. R. (2004). Use of Evolutionary Algorithms for Multi-Objective Optimization. Computational Intelligence, 20(2), 1–16.
\newline[15]Solomon, M. M. (1987). Algorithms for the Vehicle Routing and Scheduling Problems with Time Window Constraints. Operations Research, 35(2), 254-265.
\newline[16]Toth, P., \& Vigo, D. (2001). The Vehicle Routing Problem. Society for Industrial and Applied Mathematics (SIAM).
\newline[17]Garcia, S., \& Longo, E. G. (1999). Optimizing Travel Distance in the Vehicle Routing Problem. Springer.
\newline[18]Dorigo, M., Maniezzo, V., \& Colorni, A. (1996). Ant System: Optimization by a Colony of Cooperating Agents. IEEE Transactions on Systems, Man, and Cybernetics - Part B: Cybernetics, 26(1), 29–41.
\newline[19]Kennedy, J., \& Eberhart, R. C. (1995). Particle Swarm Optimization. Proceedings of the IEEE International Conference on Neural Networks, 1942–1948. https://doi.org/10.1109/ICNN.1995.488968
\newline[20] Rizzoli, Andrea-Emilio \& Oliverio, F. \& Montemanni, Roberto \& Gambardella, Luca Maria. (2004). Ant Colony Optimisation for vehicle routing problems: from theory to applications. 

